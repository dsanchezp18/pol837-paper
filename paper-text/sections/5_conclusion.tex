% Options for packages loaded elsewhere
\PassOptionsToPackage{unicode}{hyperref}
\PassOptionsToPackage{hyphens}{url}
\PassOptionsToPackage{dvipsnames,svgnames,x11names}{xcolor}
%
\documentclass[
  12pt,
  letterpaper,
  DIV=11,
  numbers=noendperiod]{scrartcl}

\usepackage{amsmath,amssymb}
\usepackage{setspace}
\usepackage{iftex}
\ifPDFTeX
  \usepackage[T1]{fontenc}
  \usepackage[utf8]{inputenc}
  \usepackage{textcomp} % provide euro and other symbols
\else % if luatex or xetex
  \usepackage{unicode-math}
  \defaultfontfeatures{Scale=MatchLowercase}
  \defaultfontfeatures[\rmfamily]{Ligatures=TeX,Scale=1}
\fi
\usepackage{lmodern}
\ifPDFTeX\else  
    % xetex/luatex font selection
  \setmainfont[]{Times New Roman}
  \setmonofont[]{Times New Roman}
  \setmathfont[]{Libertinus Math}
\fi
% Use upquote if available, for straight quotes in verbatim environments
\IfFileExists{upquote.sty}{\usepackage{upquote}}{}
\IfFileExists{microtype.sty}{% use microtype if available
  \usepackage[]{microtype}
  \UseMicrotypeSet[protrusion]{basicmath} % disable protrusion for tt fonts
}{}
\makeatletter
\@ifundefined{KOMAClassName}{% if non-KOMA class
  \IfFileExists{parskip.sty}{%
    \usepackage{parskip}
  }{% else
    \setlength{\parindent}{0pt}
    \setlength{\parskip}{6pt plus 2pt minus 1pt}}
}{% if KOMA class
  \KOMAoptions{parskip=half}}
\makeatother
\usepackage{xcolor}
\setlength{\emergencystretch}{3em} % prevent overfull lines
\setcounter{secnumdepth}{5}
% Make \paragraph and \subparagraph free-standing
\ifx\paragraph\undefined\else
  \let\oldparagraph\paragraph
  \renewcommand{\paragraph}[1]{\oldparagraph{#1}\mbox{}}
\fi
\ifx\subparagraph\undefined\else
  \let\oldsubparagraph\subparagraph
  \renewcommand{\subparagraph}[1]{\oldsubparagraph{#1}\mbox{}}
\fi


\providecommand{\tightlist}{%
  \setlength{\itemsep}{0pt}\setlength{\parskip}{0pt}}\usepackage{longtable,booktabs,array}
\usepackage{calc} % for calculating minipage widths
% Correct order of tables after \paragraph or \subparagraph
\usepackage{etoolbox}
\makeatletter
\patchcmd\longtable{\par}{\if@noskipsec\mbox{}\fi\par}{}{}
\makeatother
% Allow footnotes in longtable head/foot
\IfFileExists{footnotehyper.sty}{\usepackage{footnotehyper}}{\usepackage{footnote}}
\makesavenoteenv{longtable}
\usepackage{graphicx}
\makeatletter
\def\maxwidth{\ifdim\Gin@nat@width>\linewidth\linewidth\else\Gin@nat@width\fi}
\def\maxheight{\ifdim\Gin@nat@height>\textheight\textheight\else\Gin@nat@height\fi}
\makeatother
% Scale images if necessary, so that they will not overflow the page
% margins by default, and it is still possible to overwrite the defaults
% using explicit options in \includegraphics[width, height, ...]{}
\setkeys{Gin}{width=\maxwidth,height=\maxheight,keepaspectratio}
% Set default figure placement to htbp
\makeatletter
\def\fps@figure{htbp}
\makeatother

\usepackage{lipsum}
\newfontfamily\tfont{Times New Roman}
\addtokomafont{title}{\tfont}
\newfontfamily\sfont{Times New Roman}
\addtokomafont{section}{\sfont}
\addtokomafont{subsection}{\sfont}
\usepackage{booktabs}
\usepackage{siunitx}
\newcolumntype{d}{S[
    input-open-uncertainty=,
    input-close-uncertainty=,
    parse-numbers = false,
    table-align-text-pre=false,
    table-align-text-post=false
]}
\usepackage{amsmath,amsfonts}
\usepackage{float}
\usepackage{pdflscape}
\usepackage{csquotes}
\KOMAoption{captions}{tableheading,figureheading}
\makeatletter
\@ifpackageloaded{caption}{}{\usepackage{caption}}
\AtBeginDocument{%
\ifdefined\contentsname
  \renewcommand*\contentsname{Table of contents}
\else
  \newcommand\contentsname{Table of contents}
\fi
\ifdefined\listfigurename
  \renewcommand*\listfigurename{List of Figures}
\else
  \newcommand\listfigurename{List of Figures}
\fi
\ifdefined\listtablename
  \renewcommand*\listtablename{List of Tables}
\else
  \newcommand\listtablename{List of Tables}
\fi
\ifdefined\figurename
  \renewcommand*\figurename{Figure}
\else
  \newcommand\figurename{Figure}
\fi
\ifdefined\tablename
  \renewcommand*\tablename{Table}
\else
  \newcommand\tablename{Table}
\fi
}
\@ifpackageloaded{float}{}{\usepackage{float}}
\floatstyle{ruled}
\@ifundefined{c@chapter}{\newfloat{codelisting}{h}{lop}}{\newfloat{codelisting}{h}{lop}[chapter]}
\floatname{codelisting}{Listing}
\newcommand*\listoflistings{\listof{codelisting}{List of Listings}}
\makeatother
\makeatletter
\makeatother
\makeatletter
\@ifpackageloaded{caption}{}{\usepackage{caption}}
\@ifpackageloaded{subcaption}{}{\usepackage{subcaption}}
\makeatother
\ifLuaTeX
  \usepackage{selnolig}  % disable illegal ligatures
\fi
\usepackage{bookmark}

\IfFileExists{xurl.sty}{\usepackage{xurl}}{} % add URL line breaks if available
\urlstyle{same} % disable monospaced font for URLs
\hypersetup{
  colorlinks=true,
  linkcolor={blue},
  filecolor={Maroon},
  citecolor={Blue},
  urlcolor={Blue},
  pdfcreator={LaTeX via pandoc}}

\author{}
\date{}

\begin{document}

\setstretch{2}
\section{Conclusion}\label{conclusion}

This paper has shown that daily temperature has a significant negative
effect on presidential approval in Ecuador. Survey respondents are about
1.9 to 2.2 percentage points less likely to approve of the president
when maximum daily temperatures increase by one degree. This result is
robust to the inclusion of socioeconomic and political behaviour
controls, including variables which control for partisanship, trust in
the police, democracy, personal ideology identification, evaluations of
the economy, among others. These results are consistent with some
literature on retrospective voting and voter errors, which suggests that
voters may commit attribution errors when evaluating politician's
performance. I validate findings from Barrington-Leigh \& Behzadnejad
(2017), Lignier et al. (2023) and Quijano-Ruiz (2023), who find that
weather impacts behaviour.

I argue that the weather affects the mood of individuals negatively, and
in turn individuals search externally for factors to validate their
mood. This leads to a misattribution of mood to the president's
performance, which results in lower approval ratings. The causal
mechanism which explains these empirical findings rests on psychological
theories of mood misattribution. These describe that individuals in a
bad mood are more likely to report feelings of life dissatisfaction, and
that they are more likely to attribute their mood to external factors
(Bower, 1981; Schwarz \& Clore, 1983). I argue that warmer weather in
Ecuador may lead to a negative moods, which in turn makes citizens
direct their emotions towards the president's performance. This is
consistent with the literature on the impact of weather across a range
of outcomes, which finds that weather can have a significant impact on
behaviour (Barrington-Leigh \& Behzadnejad, 2017; Deller \& Michels,
2022; Keller et al., 2005; Lignier et al., 2023; Quijano-Ruiz, 2023).

The results also show that the effect of temperature on presidential
approval is not constant across the population. Women are more sensitive
to higher temperatures than men, also found by Quijano-Ruiz (2023) using
CPC weather data in Ecuador and by Barrington-Leigh \& Behzadnejad
(2017) in Canada. I find that the effect of temperature on presidential
approval is conditional on the region of the country and the political
ideology of the survey respondent. The result of heterogeneity across
ideological groups produces conflicting results, suggesting that minimum
and maximum temperatures have different effects on survey respondents
identifying closer to the political right. I find no difference of the
effect of temperature between those that negatively evaluate the economy
compared to those who evaluate it positively or equal relative to last
year. Results for regional heterogeneity, while understandable given
that the Amazon region is the most humid and warm region in the country,
are preliminary and should be taken with caution, because of the small
sample size of the Amazon region in the AmericasBarometer surveys.

In the same line as Quijano-Ruiz (2023), who pioneers the use of CPC
weather data in health services research, I introduce the use of this
data for political behaviour studies, with promising results. CPC
temperature data, though of lesser quality than weather station data, is
of invaluable use for countries where weather station data is not
available. There is a possibility that my temperature variables are
subject to measurement error, which could bias my results. If this is
the case, then my results are likely to be downward biased, which would
suggest that the true effect of temperature on presidential approval is
larger than what I estimate in this paper. The fact that I am able to
find statistically significant results in an observational setting
suggests that the true effect of temperature on presidential approval is
likely to be larger, and future research should aim to address this
possibility by using more precise temperature data, and by using more
sophisticated methods to address measurement error. Replicating this
study in other countries where temperature data of higher quality is
available would also be valuable, in order to validate these results and
understand the precision of CPC weather data for political science
research.

I model the effect of temperature on presidential approval in a linear
manner, which may be innacurate, given the complex nature of weather and
the behavioural responses that weather may cause. Weather likely has a
nonlinear effect on mood, which should be modeled with more
sophisticated methods in future research to more accurately understand
the effect of weather on presidential approval. I am also limited in the
way that the AmericasBarometer is collected, which is bienally, and does
not allow me to observe the effect of temperature on presidential
approval in more granular frequencies and lower levels of spatial
aggregation such as parishes or neighbourhoods.

Understanding how temperature and other weather-related variables affect
political behaviour is important for extending the literature on
attribution errors and retrospective voting, but even more so for
understanding the way that political behaviour works in Latin America, a
region which has been severely understudied in the literature. This
paper is a first step in understanding the effect of weather on
political behaviour in the literature, which moves away from the focus
on standard variables which have been proven to be influenced by factors
not present in developed countries. Understanding these mechanisms is
important for better research, but also to enact public policy for
democratic accountability.



\end{document}
